% Format
\documentclass[11pt]{article}
\usepackage[margin=1.5in]{geometry}
\pagestyle{headings}

% Packages
\usepackage{color}
\usepackage{xcolor} % Probably do not need both...
\usepackage{graphicx}
\usepackage{caption}
\usepackage{subcaption}
\usepackage[normalem]{ulem}
\usepackage{morefloats}
\usepackage[hidelinks]{hyperref}
	\hypersetup{
    colorlinks=true,
    filecolor=magenta,
    colorlinks,
    linkcolor={red!50!black},
    citecolor={blue!50!black},
    urlcolor={blue!80!black}
	}
\newtheorem{conjecture}{Conjecture}
\usepackage{comment}
\usepackage{lipsum}
\usepackage[english]{babel}
\usepackage[utf8]{inputenc}
\usepackage{amsmath, amssymb, amsthm, textcomp}
	\DeclareMathOperator*{\argmax}{arg\,max}
	\DeclareMathOperator*{\argmin}{arg\,min}
%\usepackage[bb=libus]{mathalpha}
\usepackage{bbm}
\usepackage{centernot}
\usepackage{comment}
\usepackage{mathtools}
\usepackage{mathtools}
\usepackage{mathrsfs}
\usepackage{dsfont}
\usepackage[utf8]{inputenc}
\usepackage{enumitem}
\usepackage{sectsty}
	\allsectionsfont{\mdseries\scshape} %\centering
\usepackage{diffcoeff}
% TABOUT PACKAGES
\usepackage{booktabs}
\usepackage{tabularx}
\usepackage{float}
\usepackage{caption}
\usepackage{subcaption}
\usepackage{adjustbox}
% TIKZ PACKAGES
\usepackage{tikz}
\usetikzlibrary{shapes}
\usetikzlibrary{plotmarks}
\tikzstyle{common}=[
    rectangle,
    minimum size = 3cm,
    draw=black,
    thick,
    fill=blue,
    text=white
]
\tikzstyle{specific}=[
    rectangle,
    minimum size = 1cm,
    draw=black,
    thick,
    fill=yellow,
    text=black
]
\tikzstyle{first}=[
    rectangle,
    minimum size = .75cm,
    draw=red,
    thick,
    %fill=yellow,
    text=black
]
\tikzstyle{second}=[
    rectangle,
    minimum size = .75cm,
    draw=blue,
    thick,
    %fill=red,
    text=black
]
\tikzstyle{third}=[
    rectangle,
    minimum size = .75cm,
    draw=orange,
    thick,
    %fill=green,
    text=black
]

\usetikzlibrary{decorations.pathreplacing,calligraphy}
% TO FIT LARGE TABLES
\usepackage[strict]{changepage}

% BibLaTeX
\usepackage[backend=bibtex,style=verbose-trad2]{biblatex}
%\usepackage[backend=bibtex,style=verbose-trad2]{biblatex}

\newtheorem{prop}{Proposition}

\newcommand{\sq}[1]{\langle #1 \rangle}
\newcommand{\bra}[1]{\langle #1 \vert}
\newcommand{\ket}[1]{\vert #1 \rangle}
\newcommand{\indep}{\perp \!\!\! \perp}
\newcommand{\ubrace}[1]{\underbrace{#1}}
\newcommand{\percite}[1]{\citeauthor{#1} (\citeyear{#1})}
\newcommand{\ppercite}[1]{\citeauthor{#1}, \citeyear{#1}}
\RequirePackage{bm}
\newcommand{\vect}[1]{\boldsymbol{\mathbf{#1}}}
\RequirePackage[stable]{footmisc}

\newtheorem*{assumption*}{\assumptionnumber}
\providecommand{\assumptionnumber}{}
\makeatletter
\newenvironment{assumption}[2]
 {%
  \renewcommand{\assumptionnumber}{Assumption #1-$\mathcal{#2}$}%
  \begin{assumption*}%
  \protected@edef\@currentlabel{#1-$\mathcal{#2}$}%
 }
 {%
  \end{assumption*}
 }
\makeatother


\title{\textsc{Minimum Wage and Non-Competes in a Search Model with Human Capital}}
\author{Samuel Barker \& August Bruno\footnote{The Economics
Department at the University of North Carolina.}}

\bibliography{~/Zotero/library.bib}

%\usepackage[backend=bibtex]{biblatex}
\begin{document}
%\include{sections/titlepage}
\maketitle
%----------------------------------------------------------------------------------------
%	ABSTRACT
%----------------------------------------------------------------------------------------
\begin{abstract}
\end{abstract}

\tableofcontents

\setcounter{page}{1}

\section[Model]{Model}%
\label{sec:model}

\subsection[Setting]{Setting}%
\label{sub:setting}

Workers can be employed or unemployed. We imagine firms as being a from a
partial equilibrium--i.e., match specific productivity is given by an
exogenous distribution $F(X)$. The production of a worker with human
capital $h_t \in  \{0,h\} $, where $h$ is to be estimated, matched with a
firm with match specific productivity $x$ is given by: 
\begin{align*}
    y_t = x + h_t
.\end{align*}
Wages are given by $w_t$. We specify wages as coming from some bargaining
process or else being the minimum wage, $\underline{w}$. In particular, when it is above the
minimum wage, we let the wage
$w_t$ be the wage that solves the following Nash Bargaining problem:
\begin{align*}
    \max_w \big ( V(w,h_t) - U(h_t) \big )^{\alpha} \Pi(w,X,h_t)^{1-\alpha}
,\end{align*}
where $V(w,h_t)$ is the expected value of employment at wage $w$ and
human capital $h_t$, $U(h_t)$ is the value of unemployment with human
capital  $h_t$, and $\Pi(w,X,h_t)$ is the expected profit from hiring a
worker at wage $w$ and human capital $h_t$. We explore these objects in the
latter sections, however to preempt questions, because $V(\cdot ,\cdot )$
and $\Pi(\cdot, \cdot ,\cdot)$ are quasilinear in $w$, this will provide a clean
characterization of $w_t$.


For clarity, let
\begin{align*}
    w_{\alpha} \triangleq \argmax_w \big ( V(w,h_t) - U(h_t) \big
        )^{\alpha} \Pi(w,X,h_t)^{1-\alpha}
,\end{align*}
therefore, $w_t$ will be a function of $X$ and $h_t$ given by:
\begin{align*}
    w_t(X,h_t) &= \begin{cases}
        w_{\alpha} &\text{ if } w_\alpha \ge \underline{w} \\
        \underline{w} &\text{ if } w_\alpha < \underline{w}
    \end{cases}
.\end{align*}
For simplicity of notation, however, we will often write simply $w_t$ rather than
$w_t(X,h_t)$.


\subsection[Workers]{Workers}%
\label{sub:workers}

Workers encounter firms at some Poisson rate
$\lambda_0$ when unemployed and $\lambda_1$ when employed. Worker's human
capital evolves probabilistically. While employed, a worker's human capital
can increase from $0$ to $h$ with a poisson rate of  $\xi_1$, and while
unemployed can depreciate from $h$ to $0$ with a poisson rate of $\xi_0$. A
match is destroyed at some exogenous rate $\rho$. The discount rate is
given by $r$. While unemployed, workers get flow utility of $b$. While
employed, workers simply get their wage.

\subsubsection[Unemployment]{Unemployment}%
\label{subsub:unemployment}
We can express the value of unemployment by each human capital type, $0$
and $h$.
\begin{align*}
    U(0) &= b \Delta t + \frac{1}{1+r\Delta t} \{ (\lambda_0 \Delta t
        + o(\Delta t)) \int_{S_X}^{} \max
    \{V(w_t, 0), U(0)\} dF(X) \\
         &\;\;\;\;+ (1-(\lambda_0\Delta t + o(\Delta t)) U(0)  \} \\
         U(0) \{\frac{\Delta t r}{1+r \Delta t} \} 
    &= b \Delta t + \frac{1}{1+r\Delta t} \{ (\lambda_0 \Delta t
        + o(\Delta t)) \int_{S_X}^{} \max
    \{V(w_t, 0), U(0)\} dF(X) \\
    &\;\;\;\;+ (-\lambda_0\Delta t + o(\Delta t)) U(0)  \} \\
    U(0) \{\frac{r}{1+r \Delta t} \} 
    &= b + \frac{1}{1+r\Delta t} \{ (\lambda_0
        + o(\Delta t)/\Delta t) \int_{S_X}^{} \max
    \{V(w_t, 0), U(0)\} dF(X)  \\
    &\;\;\;\;+ (-\lambda_0 + o(\Delta t)/\Delta t) U(0)  \} \\
         U(0) (r + \lambda_0)
    &= b +  \lambda_0 \int_{S_X}^{} \max
    \{V(w_t, 0), U(0)\} dF(X)
,\end{align*}
where the final line is obtained by taking the limit as $\Delta t \rightarrow
0$.


Starting with $h_t = h$ and walking through similar steps leads us to a
similar expression:
\begin{align*}
        U(h) &= \Delta t b + \frac{1}{1+r\Delta t}\{
            (\lambda_0 \Delta t +o(\Delta t)) \int_{S_X}^{} \max \{V(w_t,
            h), U(h)\} dF(X) \\
             &\;\;\;\;+ (\xi_0\Delta t +o(\Delta t)) U(0) + (1 - (\lambda_0\Delta t +
    o(\Delta t)) - (\xi_0\Delta t + o(\Delta t))) U(h) \}
,\end{align*}
which then reduces to:
\begin{align*}
    U(h)(r+\lambda_0+\xi_0) - \xi_0 U(0) &= b + \lambda_0 \int_{S_X}^{} \max \{V(w_t,
            h), U(h)\} dF(X)
.\end{align*}
Equivalently, we can write a general characterization:
\begin{align*}
    U(h_t)(r+\lambda_0+\xi_0) - \xi_0 U(0) &= b + \lambda_0 \int_{S_X}^{} \max \{V(w_t,
            h_t), U(h_t)\} dF(X)
.\end{align*}



\subsubsection[Employment]{Employment}%
\label{subsub:employment}

Everything is similar while employed, and we get the following general
expression:
%\begin{align*}
%    V(w_t,0) &= \Delta t w_t + \frac{1}{1+r\Delta t}\{
%        (\lambda_1 \Delta t +o(\Delta t)) \int_{S_X}^{} \max
%        \{V(w_t(X',0),0),
%        V(w_t,0)\} dF(X') \\
%             &\;\;\;\;+ (\xi_1\Delta t +o(\Delta t)) V(w_t,h) + (1 -
%             (\lambda_1\Delta t +
%         o(\Delta t)) - (\xi_1\Delta t + o(\Delta t))) V(w_T,0) \}
%,\end{align*}
%which reduces to:
\begin{align*}
    V(w_t,h_t)(r+\lambda_1+\xi_1+\delta) &= w_t + \lambda_1
    \int_{S_X}^{} \max \{V(w_t(X',h_t),h_t), V(w_t,h_t)\} dF(X') \\
    &\;\;\;\; + \delta U(h_t) + \xi_1 V(w_t,h)
.\end{align*}




\subsection[Firms]{Firms}%
\label{sub:firms}
We assume a free entry condition so that the value of a vacancy is zero.
Firms flow profits are given by the difference between production and wages
$X + h_t - w_t$. In the case when $h_t = h$, everything is standard as we
will show. However, when the human capital of the individual is low
($0$), there is some probability that the productivity will increase from
$0$ to $h$.
 \begin{align*}
     \Pi(w_t,X,h) &= \Delta t(X + h - w_t) + \frac{1}{1+r\Delta t} \{ (\delta
     \Delta t + o(\Delta t)) \cdot 0 \\
                  &\;\;\;\;+ (1-(\delta \Delta t + o(\Delta t))-(\lambda_1
                  \Delta t + o(\Delta t))) \Pi(w_t,X,h)\}
\end{align*}
which reduces to:
\begin{align*}
         \Pi(w_t,X,h)(r+\lambda_1+\delta) &= X + h - w_t 
.\end{align*}
In the case where $h_t = 0$, we have instead:
\begin{align*}
    \Pi(w_t,X,0) &= \Delta t(X - w_t) + \frac{1}{1+r\Delta t} \{ (\delta
        \Delta t + o(\Delta t)) \cdot 0 + (\xi_1 \Delta t + o(\Delta
        t))\Pi(w_t,X,h) \\
                 &\;\;\;\; + (1-(\delta \Delta t + o(\Delta t))
     -(\xi_1 \Delta t + o(\Delta t))
     -(\lambda_1 \Delta t + o(\Delta t)))
 \Pi(w_t,X,0)\}
\end{align*}
which reduces to:
\begin{align*}
    \Pi(w_t,X,0)(r+\lambda_1+\delta+\xi_1) = X - w_t + \xi_1 \Pi(w_t,X,h)
.\end{align*}


Again, we can write a general characterization as:
\begin{align*}
    \Pi(w_t,X,h_t)(r+\lambda_1+\delta+\xi_1) = X + h_t - w_t +  \xi_1 \Pi(w_t,X,h)
.\end{align*}


The main point of this section is to emphasis the following: when $h_t =
0$, it is possible that a firm may employ a worker at a wage  $w_t > X$
because of the expected future profits that may accrue if  $h_t$
transitions to $h$: $\xi_1 \Pi(w_t,X,h)$. This is, to our knowledge, an unexplored feature of
human capital in search models, especially in the context of minimum wages.
We also seek to extend our model to include choices by the firm to
implement non-compete clauses as a way to increase the expected future
profits associated with hiring a worker--without such a clause, a worker
will be unable to commit to a staying at the firm. With the commitment
device, however, workers and firms are able to enjoy the surplus of the
match.


If there are non-competes, we can change a number of things. For now, let a
noncompete mean the following: worker's encounter firms at a modified rate
$\lambda_1(\tau) < \lambda_1 \forall \tau$ where $\tau \in [0,T]$ is the
time since starting and $T$ is the length of the
noncompete agreements (NCA). Similarly, we have $\delta(\tau)$. Therefore,
this is a non-stationary problem.  We can write the expected profits
as follows:
 \begin{align*}
     \Pi(w_t,X,h,\tau) &= \Delta t(X + h - w_t) + \frac{1}{1+r\Delta t} \{
         ( \textstyle\int_{\tau}^{\tau + \Delta t}  \delta(t) dt + o(\Delta t)) \cdot 0 \\
                       &\;\;\;\;+ (1-(\textstyle\int_{\tau}^{\tau+\Delta t}
                       \delta(t)dt + o(\Delta
                       t))-(\textstyle\int_{\tau}^{\tau + \Delta t}
                       \lambda_1(t)dt + o(\Delta t)))
                   \Pi(w_t,X,h,\tau + \Delta t)\}
.\end{align*}
Rearranging, we get
\begin{align*}
         &\frac{(1+ r \Delta t) \Pi(w_t,X,h,\tau) - 
                   \textstyle\int_{\tau}^{\tau + \Delta t} \Pi(w_t,X,h,t)dt\}
         }{1 + r \Delta t} \\
         &\;\;\;\;\;\;\;\; = \Delta t(X + h - w_t) + \frac{1}{1+r\Delta t} \{
         ( \textstyle\int_{\tau}^{\tau + \Delta t}  \delta(t) dt + o(\Delta t)) \cdot 0 \\
         &\;\;\;\;\;\;\;\;+ (-(\textstyle\int_{\tau}^{\tau+\Delta t}
                       \delta(t)dt + o(\Delta
                       t))-(\textstyle\int_{\tau}^{\tau + \Delta t}
                       \lambda_1(t)dt + o(\Delta t)))
                   \Pi(w_t,X,h,\tau + \Delta t)\}
\end{align*}
Next we divide through by $\Delta t$. On the RHS we get,
\begin{align*}
    \frac{r \Pi(w_t,X,h,\tau)}{1 + r \Delta t}
    + 
    \frac{1}{1 + r \Delta t}\Big (\frac{\Pi(w_t,X,h,\tau) -
    \Pi(w_t,X,h,\tau + \Delta t)}{\Delta t} \Big )
,\end{align*}
and taking the limit as $\Delta t \rightarrow 0$:
\begin{align*}
    r \Pi(w_t,X,h,\tau)
    + \frac{\partial}{\partial t} \Pi(w_t,X,h,t)\Big|_{t = \tau}
\end{align*}
where the second term follows from the definition of a derivative.

The LHS is:
\begin{align*}
         &(X + h - w_t) + \frac{1}{1+r\Delta t} \{
         ( \frac{1}{\Delta t}\textstyle\int_{\tau}^{\tau + \Delta t}
         \delta(t) dt + o(\Delta t)/\Delta t) \cdot 0 \\
         &\;\;\;\; +(-(\frac{1}{\Delta t}\textstyle\int_{\tau}^{\tau+\Delta t}
                       \delta(t)dt + o(\Delta
                       t)/\Delta t)-(\frac{1}{\Delta t}\textstyle\int_{\tau}^{\tau + \Delta t}
                       \lambda_1(t)dt + o(\Delta t)/\Delta t))
                   \Pi(w_t,X,h,\tau + \Delta t)\}
.\end{align*}
Taking the limit as $\Delta t \rightarrow 0$ and using the Mean Value Theorem for Integrals, we can get:
\begin{align*}
         (X + h - w_t) - (\delta(\tau) + \lambda_1(\tau)) \Pi(w_t,X,h,\tau )
.\end{align*}
Rearranging and putting these together, we get:
\begin{align*}
    (r +\delta(\tau) +\lambda_1(\tau)) \Pi(w_t,X,h,\tau)
         &=  (X + h - w_t) - \frac{\partial}{\partial t} \Pi(w_t,X,h,t)\Big|_{t = \tau}
\end{align*}
Lastly, $\Pi(w_t,X,h,T) = \Pi(w_t,X,h)$ and $\frac{\partial }{\partial t}
\Pi(w_t,X,h,t)\Big |_{t = T} = 0$.




\subsection[Wage Determination]{Wage Determination}%
\label{sub:wage_determination}

First, we consider the case without a minimum wage for simplicity, and then
we return to that feature.

The first thing to note is that for all $h_t$,  $V(w_t,h_t)$ is increasing
in  $w_t$. Second, whenever the minimum wage is not binding, $w_t$ is
increasing in $X$.\footnote{This is obvious, but needs some justification.}

With this in mind, let $\underline{X}(h_t)$ be the match specific productivity such that
$V(w_t(\underline{X},h_t),h_t) = U(h_t)$.\footnote{Is it the case that
$\underline{X}(0) = \underline{X}(h)$? I do not think so.} With this
notation, we can simplify the value of unemployment:
\begin{align*}
    U(h_t)(r+\lambda_0+\xi_0) - \xi_0 U(0) &= b + \lambda_0
    \int_{\underline{X}(h_t)}^{\infty} V(w_t, h_t) dF(X) +
    \lambda_0F(\underline{X}(h_t)) U(h_t)
.\end{align*}


Theoretically, the result of the Nash Bargaining process could result in
some wage contract with stipulations (e.g., a non-compete) with some value
$C$. Above we explored what this would look like if the contract could be
summarized by an hourly wage rate. However workers have quasilinear
utility, therefore any contract with a value $C$ such that:
 \begin{align*}
     C_t &= V(w^{*},h_t) \\
         &\text{ where }\\
    w^{*} &=  \argmax_w \big ( V(w,h_t) - U(h_t) \big )^{\alpha} \Pi(w,X,h_t)^{1-\alpha}
\end{align*}
would be equivalent from a welfare perspective.\footnote{What edits to the
model are necessary for non-competes to have bite?} The main advantage of
the alternative contracts $C_t$ are if the standard contract $V(w^{*},h_t)$
is not feasible because $w^{*} < \underline{w}$. The firm may be able to
offer some contract $C_t$ such that $C_t = V(w^{*},h_t)$.\footnote{Note
that if $\Pi(\underline{w},X,h) < 0$, then no contract $C$ is
possible.}

In the non-compete case, $C$ would take the form of ensuring a length of
time of employment.


\end{document}
